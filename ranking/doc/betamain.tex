\documentclass{sig-alternate}
\usepackage{verbatim}
\usepackage{graphicx,subfigure,multirow}
\usepackage{epstopdf}
\usepackage{comment}
\usepackage{array}
\usepackage{mathtools}
\usepackage{amsmath}
\usepackage{color}
\usepackage{times}
\usepackage{multirow}
\usepackage{balance}
\usepackage{url}
\usepackage{eurosym}

% from JK
\usepackage{caption2}
\usepackage{pgfplots}
\usepackage{url}
\usepackage{tikz}


\makeatletter
\newif\if@restonecol
\makeatother
\let\algorithm\relax
\let\endalgorithm\relax
\usepackage[vlined, boxed,linesnumbered]{algorithm2e}
%p

\newcommand{\tabincell}[2]{\begin{tabular}{@{}#1@{}}#2\end{tabular}}
\newcommand{\tc}[1]{\multicolumn{1}{c}{#1}}


\newcommand{\agp}[1]{\noindent{\textcolor{red}{Aditya: #1}}}


\newcommand{\reminder}[1]{\textbf{[** #1 **]}}  % to fix
\newcommand{\hide}[1]{} %hide
\newcommand{\vpara}[1]{\vspace{0.05in}\noindent\textbf{#1 }}
\newcommand{\para}[1]{\vspace{0.01in}\noindent\textbf{#1 }}
\newcommand{\secref}[1]{Section~\ref{#1}} %section reference
\newcommand{\Real}{\ensuremath{\mathbb{R}}}  % Real numbers
\newcommand{\figref}[1]{Figure~\ref{#1}} %section reference
\newcommand{\beq}[1]{\begin{equation}#1\end{equation}}
\newcommand{\beqn}[1]{\begin{eqnarray}#1\end{eqnarray}}
\newcommand{\beal}[1]{\begin{align}#1\end{align}}
\newcommand{\besp}[1]{\begin{split}#1\end{split}}
\DeclareMathOperator*{\argmin}{arg\,min}
\DeclareMathOperator*{\argmax}{arg\,max}

% from JH
\newcommand{\from}[2]{{\bf [{\sc from #1:} #2]}}
\newcommand{\eg}{{\sl e.g.}}
\newcommand{\ie}{{\sl i.e.}}
\newcommand{\etc}{{\sl etc.}}
\newcommand{\etal}{{\sl et al.}}
\newcommand{\wrt}{{\sl w.r.t.}}
%\newcommand{\omitmark}{{... \sl[Omitted]}}

\def\bb{{\mathbf{b}}}
\def\bx{{\mathbf{x}}}
\def\by{{\mathbf{y}}}
\def\cY{{\mathcal{Y}}}
\def\RR{{\mathbb{R}}}


\hide{
% Package to generate and customize Algorithm as per ACM style
\usepackage[ruled]{algorithm2e}

\renewcommand{\algorithmcfname}{ALGORITHM}
\SetAlFnt{\small}
\SetAlCapFnt{\small}
\SetAlCapNameFnt{\small}
\SetAlCapHSkip{0pt}
\IncMargin{-\parindent}
}

%\newcommand{\eeq}[1]{\end{equation}\normalsize}

\newtheorem{definition}{Definition}
\newtheorem{problem}{Problem}
\makeatletter
\renewcommand*{\@opargbegintheorem}[3]{\trivlist
  \item[\hspace{2mm} \hskip \labelsep{\scshape #1\ #2}] \hspace{1mm} \textsc{(#3).}\ \itshape}
\makeatother

\newdef{theorem}{Theorem}
\newtheorem{hypothesis}{Hypothesis}
\newtheorem{assumption}{Assumption}



\newcommand{\bm}[1]{\textcolor{blue}{#1}}





\begin{document}

\title{
Debiasing Crowdsourced Batches % Annotation \\ on Batches of Data Items
}



\numberofauthors{1}
\author{
\alignauthor Honglei Zhuang, Aditya Parameswaran, Dan Roth, Jiawei Han \\
\affaddr{Department of Computer Science}  \\
\affaddr{University of Illinois at Urbana-Champaign}  \\
\email{\{hzhuang3, adityagp, danr, hanj\}@illinois.edu}
}
% \alignauthor Jiawei Han \\
% \affaddr{University of Illinois at Urbana-Champaign}  \\
% \email{ hanj@illinois.edu}



%
% Put no more than the first THREE authors in the \author commandpp

% NOTE: All authors should be on the first page. For instructions
% for more than 3 authors, see:
% http://www.acm.org/sigs/pubs/proceed/sigfaq.htm#a18


\maketitle


\begin{abstract}

Crowdsourcing is the de-facto standard for gathering annotated data.
While, in theory,
data annotation tasks are assumed to be attempted by workers independently,
in practice,
data annotation tasks are often grouped into batches to be presented
and annotated by workers together,
in order to save on the time or cost overhead of providing instructions or necessary background.  
Thus, even though independence is usually assumed between annotations on data items within the same batch,
in most cases, a worker's judgment on a data item can still be affected by other data items within the batch, leading to additional errors in collected labels.
%Although annotation bias of individual data items and sequential data items has been explored,
%little research has been done on batches of data items.
%Grouping data items in batches can facilitate crowdsourced data annotation of a great number of tasks
% (\eg~clustering, outlier detection~\etc).
In this paper, we study the data annotation bias when data items are presented as batches
to be judged by workers simultaneously.
We propose a novel worker model to characterize the annotating behavior on data batches,
and present how to train the worker model on annotation data sets.
We also present a debiasing technique to remove the effect of such annotation bias
from adversely affecting the accuracy of labels obtained.
Our experimental results on synthetic and real-world data sets
demonstrate that our proposed method can achieve up to +57\% improvement in $F_1$-score 
compared to the standard majority voting baseline.  
%and up to +14-17\% improvement comparing to a tuned majority voting strategy.  

%the effectiveness of our proposed method.

\hide{
Data annotation bias is found in many situations.  Often it can be ignored
as just another component of the noise floor.  However, it is especially
prevalent in crowdsourcing tasks and must be actively managed.  Annotation
bias on single data items
has been studied with regard to data difficulty, annotator bias, etc., while
annotation bias on batches of multiple data items simultaneously presented to
annotators has not been studied.  In this paper, we verify the existence of
``in-batch annotation bias'' between data items
in the same batch.  We propose a factor graph based batch annotation model to
quantitatively capture the in-batch annotation bias, and measure the bias during a
crowdsourcing annotation process of inappropriate comments in LinkedIn.  We
discover that annotators tend to make polarized annotations for the entire batch
of data items in our task.  We further leverage the batch annotation model to propose a
novel batch active learning algorithm.  We test the algorithm on a real
crowdsourcing platform and find that it outperforms in-batch bias na\"{\i}ve
algorithms.
}

\end{abstract}



% A category with only the three required fields
\category{H.2.8}{Database Applications}{Data mining}
%\category{H.3.3}{Information Search and Retrieval}{Text Mining}
\category{J.4}{Social and Behavioral Sciences}{Psychology}
%\category{H.4.m}{Information Systems}{Miscellaneous}
%\terms{Algorithms, Experimentation}

\keywords{Crowdsourcing; annotation bias; worker model}



\sloppy

\section{Introduction}
\label{sec:intro}

Crowdsourcing provides an inexpensive and efficient method to annotate data for various machine learning tasks 
by employing massive workforce from global Internet users.  
Popular online crowdsourcing service operators include 
Amazon Mechanical Turk\footnote{\url{https://www.mturk.com/}} and CrowdFlower\footnote{\url{http://www.crowdflower.com/}}.  
However, in contrast to expert annotated data, 
crowdsourced annotation usually suffers from relatively higher level of bias.  
Erroneous annotated data might lead to vulnerable performance of trained classifiers 
or imprecise evaluation of tested models, 
which limits the adoption of crowdsourcing as a regular method of annotated data collection.  

Data annotation bias has been studiend in multiple settings.  
Modeling workers' annotating bias on single data items has been studied in a number of studies~\cite{raykar:nips2011ranking,raykar:icml2009,raykar:jmlr2010,whitehill:nips2009}.  
In this scenario, data items are presented separately to the workers 
and no assumption is made on the interference between judgments on different data items.  
We refer to this setting as the \emph{independent judgments}.  
Recently, data items annotated in sequence also attracted research attention~\cite{mozer:nips2010,scholer:sigir2013,scholer:sigir2011}.  
Observing that a worker usually makes judgments on a sequence of data items, 
to some extent there could be dependencies between judgments made on consecutively presented data items.  
This setting could be referred to as \emph{sequential judgments}.  

In this paper, we study another interesting scenario when data items are organized into (small) a batch 
and presented to a worker to be judged together.  
We refer to this setting as \emph{batch judgments}.  
We present a graphical model example of these three different scenarios, as shown in Figure~\ref{fig:worker_mode}.  
Circles represent random variables $y_i'$, which models the annotation given by workers on the $i$-th data item, 
while black boxes represent certain kinds of dependencies.  
Figure~\ref{subfig:mode_ind} shows the scenario when workers make judgments on each data item independently.  
This usually applies when data items are presented to workers separately, 
or when there are rarely correlations between data items.  
Figure~\ref{subfig:mode_seq} as an example of sequential judgments, 
illustrates that there might be dependencies between judgments made on two consecutive data items.  
This case often happens when data items are presented to workers in a sequence, 
and there are some certain correlations between consecutive data items.  
For example, judging the relevance of a list of documents to a fixed query.  
Figure~\ref{subfig:mode_batch} shows the scenario we study in this paper, 
namely when judgments are made simultaneously to a batch of data items.  
This scenario usually happens when workers are required to make judgments to several data items at the same time, 
or the data items in the same batch are strongly connected.  
Notice that the size of batch is usually relatively small (\eg~$\leq 5$) 
because a worker usually cannot pay attention to more data items at the same time.  
When the batch becomes too large, the scenario might be reduced to the case when workers make sequential judgments.  



This judging-in-batch setting has been noticed and adopted in crowdsourcing practices of 
object recognition~\cite{su:aaai2012}, clustering~\cite{gomes:nips2011}, and sorting~\cite{marcus:vldb2011}, 
which facilitates some tasks requiring workers to judge data items based on comparison, 
also benefits other tasks on efficiency and cost.  
However, little research has explicitly addressed this setting and studied the annotation bias.  
Our previous research~\cite{zhuang:wsdm2015} also noticed this specific type of annotation bias, 
but instead of focusing on debiasing, 
we exploited the bias to develop an active learning algorithm aiming to improve a certain classifier performance.  



\begin{figure}[!t]
  \centering
  \subfigure[Independent judgments]{
    \label{subfig:mode_ind}
    \includegraphics[width=0.95\columnwidth]{figures/mode_ind}
  }\\
  \vspace{0.2in}
  \subfigure[Sequential judgments]{
    \label{subfig:mode_seq}
    \includegraphics[width=0.95\columnwidth]{figures/mode_seq}
  }\\
  \vspace{0.2in}
  \subfigure[Batch judgments]{
    \label{subfig:mode_batch}
    \includegraphics[width=0.95\columnwidth]{figures/mode_batch}
  }
  \caption{\label{fig:worker_mode}
  Graphical model examples of three different scenarios of crowdsourced annotation: 
  independent judgments usually occur with interfaces where data items presented separately; 
  sequential judgments apply to the case when data items are presented in a long sequence;
  batch judgments usually apply to the scenario when data items are presented in a relatively small batch.  
  Circles of $y_i'$ denote random variables representing the annotation given by workers, 
  while black boxes are factor functions modeling the dependencies.  
  }
\end{figure}

There are several research challenges in studying this problem.  
First, how to model workers' behavior when they make judgments in batches?  
Second, how to leverage the model to debias the crowdsourced annotation of data batches?
The following contributions are made regarding these questions:
\begin{enumerate}
  \item \emph{Proposing an interpretable worker annotation model on small batches of data.}
        We propose a novel worker model for binary annotating behavior with data items organized in small batches.  
        The model incorporates independent judgments and batch judgments based on ranking.  
        Different from the factor graph model in our previous work~\cite{zhuang:wsdm2015}, 
        our novel model is more interpretable.  
  \item \emph{Debiasing annotation data obtained as batches.}
        Based on our proposed worker model, we provide an algorithm to debias the inferred labels 
        when they are collected from data items in small batches.  
  \item \emph{Conducting experiments on real-world crowdsourcing platform.}  
        We conduct experiments on both synthetic and real-world crowdsourcing data sets 
        to verify the effectiveness of our proposed model and debiasing strategies.  
        Experimental results show the effectiveness of our debiasing method over other baselines.   
        %the F1-score of the inferred labels on the real data set can be raised from 88\% to 90\%.  
\end{enumerate}

The rest of this paper is organized as follows:
Section~\ref{sec:prelim} introduces the background of crowdsourcing, and formalizes the research problem;
Section~\ref{sec:worker} proposes the worker model for annotating small batches of data;
Section~\ref{sec:debias} presents a strategy to debias batch annotations;
Section~\ref{sec:exp} illustrates experimental results;
Section~\ref{sec:related} introduces related work and Section~\ref{sec:conclusion} concludes. 







%!TEX root=betamain.tex

%\vspace{0.2in}
\section{Preliminaries}
\label{sec:prelim}
%



\hide{
We start by introducing the background of a crowdsourcing platform, CrowdFlower.
Then we define the notations used in this paper and formalize the research problem we study.


\subsection{Crowdsourcing Platform}

We briefly introduce the background of an online crowdsourcing platform, CrowdFlower\footnote{\url{http://www.crowdflower.com/}}.
On CrowdFlower, users can design and submit a job to the online platform.
The platform will automatically assign the job to global crowd workers (termed as ``contributors'' in CrowdFlower).
Each ``job'' submitted consists of a data set, an instruction of how the data set should be labeled
and the interface designed for workers to interact with the platform.
A data set contains several ``units'', and each unit may be assigned to multiple workers to work on.
Each worker can generate a ``judgment'' for each unit.
Units are the minimum piece of work to be priced and assigned to workers.
However, for some tasks the minimum unit price (\ie~0.01 USD) might be too high,
and it would be extremely inefficient for workers to judge only a data item in a unit,
especially when the judgment needs to be done based on some context that potentially may be shared by different data items.
For example, in the task of identifying whether a document is relevant to a certain query,
if each unit is designed to be a pair of document and query,
a worker has to read both the query and the document to make a judgment;
in contrast, if a unit is designed to be a query and a batch of documents,
the worker can read the query once, and make several judgments on multiple documents within a query.
Thereby workers can save time,
and users can lower their cost of data annotation.


For each job, the user can require all the workers to go through a certain number of ``test questions''
before they actually start working on the job.
Test questions are units of which the correct judgment is already known.
Therefore one can evaluate the accuracy of each worker.
Usually workers with an accuracy lower than 70\% on test questions are not allowed to work on the job.
We believe that these test questions can be better utilized rather than simply calculating the overall accuracy.
Richer information about worker behavior might be hidden within the judgments of test questions.
Users can also insert some units with known answers intermingled with other units while workers work on the job,
to monitor the real time accuracy of each worker.
Similarly, if the accuracy of a worker drops to lower than 70\%, she will be forbidden from working on the rest of the job.

After the judgments from (multiple) workers toward a unit is obtained,
one can aggregate these judgments by a majority voting strategy to determine the final annotation of this unit.
However, this strategy may not be the best strategy due to the possible annotation bias when each unit contains multiple data items.
We can also develop other aggregating strategies to apply on the judgments,
which better recovers the true labels of data items.
}

%

In this section, we formally define the concepts and notations we use in this paper;
we then formalize the problem of debiasing crowdsourced batches.


\subsection{Basic Concepts and Terminology}

First we need to formalize several basic concepts in a crowdsourcing platform.
Suppose we are given a set of data items $X = \{x_i\}$, where $i=1, \ldots, n$.
Each data item is associated with a label $y_i \in \mathcal{Y}$, 
and we thereby define $Y = \{y_i\}_{i=1}^n$.
In following discussion, we focus on a binary classification task, 
where $\mathcal{Y} = \{0, 1\}$, 
but our framework generalizes to multi-class or rating cases seamlessly (Cf. Section~\ref{sec:ext}).  
According to a standard formalization in learning theory for binary classification, 
we suppose each $(x_i, y_i)$ is generated from a joint probability distribution $P_{\mathcal{X} \mathcal{Y}}$.
We define an inherent score $\eta_{x_i}$ to be the conditional probability $P(y_i = 1 | x_i)$.  
For simplicity, we denote the inherent score as $\eta_i$.  

In a job or task submitted to a crowdsourcing platform,
we can assemble several data items into a batch. %  (``unit'' in CrowdFlower terminology).p
Each batch $\bb_j$ is represented by a set of indices of data items in the batch,
denoted as $\{b_{j1}, \ldots, b_{jk}\}$, 
where $k$ is the size of a batch. 
To be strict, data items in the batch should be represented by $\bx_j = \{x_{b_{j1}}, \ldots, x_{b_{jk}}\}$.  
However, for simplicity, we denote data items in the batch specified by $\bb_j$ as $\{x_{j1}, \ldots, x_{jk}\}$.  
Similarly, we define $\by_j = \{y_{j1}, \ldots, y_{jk}\}$ to be true labels associated with data items in $\bx_j$, 
where $y_{jl}$ is the true label of $x_{jl}$ according to $Y$, $\forall 1 \leq l \leq k$.  
In CrowdFlower language, 
a batch corresponds to a single ``unit'', 
% several data items are assembled into a single ``unit'',pp
where a worker has to judge the entire unit at the same time;
in Mechanical Turk language,
a batch corresponds to a single ``HIT'' (short for Human Intelligence Task).
Usually, data items in the same batch might share the same context, background, or the same instruction, 
in order to reduce the overhead.  
For example, if one is asked to judge whether 
a review about a restaurant is positive or negative, 
it might save time for workers  
by grouping reviews of the same restaurant into the same batch, 
as they only need to read the description of the restaurant once 
before they can make multiple judgments on different reviews.  

\hide{
    Notice that we are only concerned with the scenario when data items are formed into \emph{small batches}.
    Depending on different format of data items, a small batch can have different scale of size.
    For example, in inappropriate comments identification task, we basically confine $k \leq 5$.
}

%

As we assemble data items into batches, 
each worker has to judge the entire batch as a single judgment.  
Given a batch $\bb_j$, the judgment provided by a worker can be represented as $\by_j' = \{y_{j1}', \ldots, y_{jk}'\}$, 
where $y_{jl}' \in \mathcal{Y}$ is the annotation of data item corresponding to 
$x_{jl}$, provided by the worker.
Noting that the worker annotation $\by_j'$ can be different from the true label $\by_j$.  
We refer to worker annotation as ``\emph{annotation}'', 
while the ground-truth label is referred to as simply the ``\emph{label}''.  

In CrowdFlower, as a judgment can only be made based on a unit,
workers are not allowed to submit partial results on a batch 
(as with Mechanical Turk).
However, one can always add an ``unknown'' option for every data item,
so that the workers can provide partial results on a batch.  
For simplicity, we consider no partial judgments in the rest of the paper.  
% Our model can also be easily adapted to the scenario with partial judgment on a batch.
%We do not allow users to provide partial feedback as it complicates the pricing system.  

Now, we are in a position to give a formal definition for a batch of data items:
\begin{definition}
[Batch]
Given a data set $(X, Y)$,
a batch of data items with size $k$ extracted from the given data set can be represented as $(\bb_j, \bx_j, \by_j, \by_j')$, 
where $\bb_j = (b_{j1}, \ldots, b_{jk})$ is a set of indices for $X$ and $Y$; 
$\bx_j = \{x_{j1}, \ldots, x_{jk}\}$ is a set of all the data items, indexed by $\bb_j$; % (\ie~$x_{jl} = x_{b_{jl}}$); 
$\by_j = \{y_{j1}, \ldots, y_{jk}\}$ consists of the corresponding true labels of data items in $\bx_j$; %, also indexed by $\bb_j$; 
$\by_j' = \{y_{j1}', \ldots, y_{jk}'\}$ is the worker annotation on the set of the batch.  
\end{definition}

%p
\noindent Additionally, a set of batches can be defined as:

\begin{definition}
Given a data set $(X, Y)$, 
a set of batches extracted from the given data set is denoted as $\cA = (B, X_B, Y_B, Y_B')$, 
where $B=\{\bb_j\}_{j=1}^m$ consists of the indices of each batch; 
$X_B=\{\bx_j\}_{j=1}^m$ is the set of data item batches, 
with their corresponding true labels $Y_B=\{\by_j\}_{j=1}^m$ 
and worker annotations $Y_B'=\{\by_j'\}_{j=1}^m$.  
\end{definition}

\vpara{Remarks.} 
1) Notice that a data item $x_i \in X$ may certainly appear in multiple batches in $\cA$.
That is, $x_{jl}$ and $x_{j'l'}$ may refer to the same data item as long as $b_{jl} = b_{j'l'}$;
% If data items in two different batches share the same 
% index as indicated by corresponding item in $\bb_j$, 
% they refer to the same data item in $X$;  
2) For the sake of fully utilizing the workforce of crowds, 
without loss of generality, we focus on the scenario when all batches have the identical size $k$. 
However, our model generalizes to the case when batches have different sizes;    
3) In some real world crowdsourcing platforms, a batch can actually be judged by multiple workers,
which means there could be multiple $\by_j'$'s associated to a single $(\bb_j, \bx_j, \by_j)$--- for instance, this is referred to as multiple {\em assignments} on Mechanical Turk.  
However, for the purposes of debiasing, 
it is equivalent to regard a single batch as multiple batches with identical $(\bb_j, \bx_j, \by_j)$
but associated with judgments made by different workers $\by_j'$. 

%This does not affect our proposed model.  pp


\hide{
Denote all the batches as a set of batches $B = \{b_j\}$ where $j = 1, \ldots, m$.
Notice that a data item can appear in multiple batches in $B$.
For the sake of fully utilize the workforce of crowds,
without loss of generality,
we only consider the scenario when all the batches in $B$ have the identical size $k$.

We denote all the judgments on a batch set $B$ as $Y'$.
In real world crowdsourcing platform, a batch can actually be judged by multiple workers, 
which means there could be multiply $y_j'$'s associated to a single batch $b_j$.  
However, this is equivalent to regard a single batch $b_j$ as multiple identical batches with different indices, 
and assign each batch with a unique judgment made by different workers.  
This does not affect the effectiveness of our proposed model.  
}


\hide{
Now we formalize the definition of a data set:
\begin{definition}
[Data Set of Batches]
A data set of batches consists of $(X_B, Y_B)$,
where $B = \{b_j\}_{j=1}^m$ is a set of batches with each $b_j = \{b_{j1}, \ldots, b_{jk}\}$ indicating a set of indices;
each batch $x_j \in X_B$ is a set of data items $\{x_{j1}, \ldots, x_{jk}\}$ indexed by $b_j$,
and each batch $y_j \in Y_B$ indicates the corresponding labels $\{y_{j1}, \ldots, y_{jk}\}$.
\end{definition}
}


%We can simply regard them as judgments on multiple identical batches indexed differently, but with the same data items.

\subsection{Problem Definition}

Based on the concepts described thus far, 
we can formalize the problem of debiasing crowdsourced batches as the following:

\begin{problem} [Debiasing Crowdsourced Batches]
Suppose we have a labeled data set $(X_L, Y_L)$ with $Y_L$ known, 
as well as its extracted batches and their crowdsourced annotation $(B_L, X_{B_L}, Y_{B_L}, Y_{B_L}')$.  
If we are then given another unlabeled data set $X_U$, 
as well as its extracted batches and crowdsourced annotation $(B_U, X_{B_U}, Y_{B_U}')$, 
the objective is to infer the true labels $Y_{U}$ associated with $X_{U}$ from the crowdsourced annotation.  
\end{problem}
Notice that our problem formulation as described above requires
as input labeled and annotated data items for training purposes. 
In practice, the labeled data for training 
can be collected from the ``test questions'' 
with ground-truth labels, 
inserted by the crowdsourcing platform for the 
purpose of quality control and monitoring
of workers.  
The usage of test questions is standard practice: 
As an example, in CrowdFlower, all workers have to attempt a 
certain number of test questions with correct labels 
and need to achieve an accuracy over a certain threshold (\eg~70\%) 
before they can proceed to work on the regular task(s).  
Also, additional hidden data items with known labels 
can be inserted into the regular tasks
to monitor the accuracy of workers.  
In our setting, worker behavior on these
test questions or labeled data can additionally be used
for training purposes.


Also notice that in this version of our problem
formulation, we assume identical worker behavior.  
This is a more standard setting in crowdsourcing practice 
as there is usually not enough work done by each worker to ascertain individual behavior.
Also, it is straightforward to extend our model when 
different workers have different behavior when working on tasks.
% \agp{can we say something like this:
% In practice, the version where worker behavior is
% assumed to be identical is also more standard because
% often we don't have enough work done by each worker to
% ascertain individual worker behavior.} 



\hide{
If we do want to model the worker behavior 
This assumption is not always true, 
but in order to reduce the 
but considering the data sparsity in real world crowdsourcing scenario 
(each worker usually answers only 10 batches test questions),
building an accurate enough worker model for each worker is not realistic.
}
%ppp



\begin{table}[!t]
\centering
 {\caption{Notation description.}\label{tab:notation}}
{
  \begin{tabular}{@{}c@{}||p{6.8cm}}
  \hline    
  Notation & Description \\ \hline \hline
  $X$      & Set of all the data items $\{x_i\}_{i=1}^n$  \\ \hline
  $Y$      & Set of all true labels associated with data items in $X$ \\ \hline
  $\bb_j$    & A set of data item indices $\{b_{jl}\}_{l=1}^{k}$ \\ \hline
  \multirow{2}{*}{$\bx_j$}    & A data item batch $\{x_{jl}\}_{l=1}^{k}$ where $x_{jl}$ is extracted from the data item in $X$ with index specified by $b_{jl}$ \\ \hline
  \multirow{2}{*}{$\by_j$}    & A label batch consists of true labels $\{y_{jl}\}_{l=1}^{k}$ associated with data items in $\bx_j$ \\ \hline
  \multirow{2}{*}{$\by_j'$}   & Worker annotation collected from a crowdsourcing platform for data items in $\bx_j$ \\ \hline
  $B$      & Set of all the batches $\{\bb_j\}_{i=1}^m$ \\ \hline
  $X_B$    & Set of all the data item batches \\ \hline
  \multirow{2}{*}{$Y_B$}    & Set of all the true labels associated with data item batches in $X_B$ \\ \hline
  \multirow{2}{*}{$Y_B'$}   & Set of all the worker annotation from crowds on data item batches in $X_B$ \\ \hline
  \end{tabular}
}
%\vspace{-0.3in}
\end{table}

\section{Worker Annotation Model On Small Batch of Data}
\label{sec:worker}

In this section, we first describe our model for workers' annotation behavior on a small batch of data;
then we introduce how to train the model based on a training data set.  
%finally we presents the parameters of this model learned from a real world data set.  

The basic idea is, when a worker judges a batch of data items, 
she can either choose to judge data items independently as if they are presented alone,
or to rank all the data items according to their relative relations 
and annotate the top several items as positive, leaving the rest in the batch as negative.  


\vpara{Plackett-Luce model.}
Before we delve in our model, we first recap a probability model for generating rankings 
based on scores associated with items, namely the Plackett-Luce model~\cite{luce:2005, plackett:1975}.  
Without loss of generality, suppose we are given a set of items $x_1, \ldots, x_k$.
Each item $x_i$ is associated with a certain score $s(x_i) > 0$.  
A ranking of these items can be represented as a bijection $\pi : \{i\}_{i = 1}^k \mapsto \{x_i\}_{i = 1}^k$, 
which maps the $i$-th position in the ranking to the item at this position.  
The corresponding ranking list can be represented as 
$\pi(1) \succ \cdots \succ \pi(k)$. 
In Plackett-Luce model, the probability of generating a ranking $\pi$ is:
\beal{
P(\pi) = \prod_{i = 1}^k \frac{s(x_i)}{\sum_{r = i}^{k} s(x_r)}	
}%
It can be interpreted as the following process:
Initially we have a pool $A$ of all the data items.  
Each time one picks an item $x_i$ from a pool $A$ of data items with a probability proportional to its score, namely:
\beal{
P(\text{picking $x_i$ from $A$}) = \frac{s(x_i)}{\sum_{x_r \in A} s(x_r)} \nonumber
}
This item is then removed from the pool $A$ and placed at the next position in the ranking.  
Repeat this operation until $A$ becomes empty.  
The probability of generating a ranking list according to this process is equivalent to the probability described in the Plackett-Luce model.  


\vpara{Worker model.}
Now we introduce our worker model for annotating small batches of data items.  
Again, without loss of generality, suppose we are given a batch $b_j$ of $(x_1, \ldots, x_k)$, namely $b_{ji} = i$.  
Also, recall that for each data item $x_i$, we denote its predictive probability $P(y_i = 1 | x_i)$ as $\eta_i$.  
Although this value is not explicitly known, 
we assume the worker has the ability to estimate this value based on the features of data item.  

When a worker starts to work on this batch of data items, 
with a certain probability $0 < \lambda < 1$, 
the worker chooses to judge each data item $x_i$ independently based on the $\eta_i$ value.  
More concretely, the worker generates $y_i' = 1$ with probability $\eta_i$ and $y_i' = 0$ with probability $(1 - \eta_i)$.  
With probability $(1 - \lambda)$, the worker chooses to first rank all the data items based on their probability of being positive, 
then annotates the top-$\tau$ items in the ranking as positive, leaving the other items annotated as negative.  
To be precise, the worker generates a ranking $\pi$ for $k$ items in the batch according to the Plackett-Luce model, 
with the scoring function defined as $s(x_i) = \eta_i$.  
Then the worker draws an integer $0 \leq \tau \leq k$ from a certain distribution, 
where $p_\tau$ denotes the probability of drawing the integer $\tau$.    
For data items $x_i \in \{\pi^{-1}(1), \ldots, \pi^{-1}(p)\}$ (could be empty if $p = 0$), the worker annotates them as $y_i'=1$, 
while other data items not within the top-$p$ of the ranking $\pi$ are annotated as $y_i' = 0$.

The intuition of this model is to capture two behavior pattern of workers.  
Sometimes workers use an absolute standard to judge data items from different batches, 
and therefore remain consistent performance.  
Nevertheless, sometimes workers might judge data items within a batch by comparison.  
They have an expectation of data distribution, 
which is reflected by the distribution of generating $p$, 
as it characterizes the probability of having $p$ positives within $k$ data items.   
They still have the ability to judge the relative relationships between data items in the same batch,
which is captured by the Plackett-Luce model for generating the ranking, 
but when they try to apply their expectation of data distribution on the batch, 
bias might occur.  


\vpara{Model learning.}
Parameters need to be determined in this worker model is:
the probability of making independent judgments $\lambda$, 
and the distribution of the number of positive annotation when making relative judgments, 
represented by $p_0, \ldots, p_k$, where $0 \leq p_{\tau} \leq 1$ and $\sum p_{\tau} = 1$.  

Suppose we are given a set of $n_L$ items $X_L$ with their true labels $Y_L$ known.  
Then we form them into $m_L$ batches $B_L$, send them to the crowds, 
and obtain their annotation from workers, denoted as $Y_L'$.  


For simplicity, we write $x_{b_{jt}}$ as $x_{jt}$, and similar to $y_{jt}$ and $\eta_{jt}$.
For each batch $b_j \in B_L$, we denote the set of items annotated by workers as positive as $X_{j}^1 = \{x_{jt} | y_{jt}' = 1\}$, 
and the set of items annotated as negative as $X_{j}^0 = \{x_{jt} | y_{jt}' = 0\}$.  


We train the model by maximum likelihood estimation.  
The likelihood of the obtained annotation can be written as:
\beal{
	L = \prod_{j=1}^{m_L} \biggl[ 
		\lambda \prod_{t=1}^{k} \eta_{jt}^{y_{jt}} (1 - \eta_{jt})^{(1 - y_{jt})} 
		%\prod_{x_{jt1} \in X_{j}^1} \eta_{jt1} \prod_{x_{jt0} \in X_{j}^0} (1 - \eta_{jt0}) \nonumber \\
		+ (1 - \lambda) p_{\tau_j} P(X_{j}^1 \succ X_{j}^0)  
	\biggr] 
}%
where $\tau_j = |X_{j}^1|$ is the number of positive annotation in batch $b_j$;
$P(X_{j}^1 \succ X_{j}^0)$ denotes the probability of generating any rankings $\pi$
that rank items in $X_{j}^1$ higher than any items in $X_{j}^0$, 
namely:
\beal{
	P(X_{j}^1 \succ X_{j}^0) = \sum_{\pi \in R(X_{j}^1, X_{j}^0)} P(\pi) \nonumber 
}%
where $R(X_1, X_0) = \{ \pi |\pi^{-1}(x_{0}) > \pi^{-1}(x_{1}),  \forall x_{1} \in X_1, x_0 \in X_0 \}$;
and $P(\pi)$ is defined by the Plackett-Luce model.  

Applying an EM-algorithm, where at E-step, we can have 
\beal{
	\hat{\lambda}_j = \frac{\hat{\lambda} \prod_{t=1}^{k} \eta_{jt}^{y_{jt}} (1 - \eta_{jt})^{(1 - y_{jt})}}
	{\hat{\lambda} \prod_{t=1}^{k} \eta_{jt}^{y_{jt}} (1 - \eta_{jt})^{(1 - y_{jt})} + 
	(1 - \hat{\lambda}) \hat{p}_{\tau_j} P(X_{j}^1 \succ X_{j}^0)} \nonumber 
}

And at M-step, we update the parameters $\hat{\lambda}$ and $\hat{p}_{\tau}$ by
\beal{
	\hat{\lambda} = \sum_{j=1}^{m_L} \hat{\lambda}_j, \hspace{0.2in}
	\hat{p}_{\tau} = \frac{1}{\hat{Z}} \sum_{j=1}^{m_L} (1 - \hat{\lambda}_j) \mathbf{1}_{\{ |X_{j}^1| = \tau\}} \nonumber 
}%
where $\hat{Z} = \sum_{j=1}^{m_L} (1 - \hat{\lambda}_j)$.  




\section{Debiasing Annotation}
\label{sec:debias}

In this section, we introduce given the trained worker model, 
how to debias annotation collected for small batches of data.  
More precisely, given a set of $n_U$ unlabeled data items $X_U$,
assemble them into $m_U$ batches represented by $B_U$,  
as well as their annotation obtained from the crowds $Y_U'$,
how to infer their true labels $Y_U$.  

The basic idea is, based on the given worker model,
try to infer $\eta_i$ for each $x_i \in X_U$,
then we simply apply the Bayes classifier to determine the inferred label, 
which yields $\hat{y}_i = 1$ if $\eta_i > 0.5$, or $\hat{y}_i = 0$ if $\eta_i \leq 0.5$.  

We again adopt a maximum likelihood estimate.  
The log-likelihood of the obtained annotation is:
\beal{
	\log L (\eta) = \sum_{j=1}^{m_U} 
	\log \biggl[ 
		&\hat{\lambda} %\prod_{t=1}^{k} \eta_{jt}^{y_{jt}} (1 - \eta_{jt})^{(1 - y_{jt})} 
		\prod_{x_{jt1} \in X_{j}^1} \eta_{jt1} \prod_{x_{jt0} \in X_{j}^0} (1 - \eta_{jt0}) \nonumber \\
		+ &(1 - \hat{\lambda}) \hat{p}_{\tau_j} P(X_{j}^1 \succ X_{j}^0)  
	\biggr] 
}%

Notice that $\hat{\lambda}$ and $\hat{p}_{\tau_j}$ are parameters learned from Section~\ref{sec:worker},
and $P(X_{j}^1 \succ X_{j}^0)$ is also a function of $\eta_{i}$'s.  
Similarly, we apply an EM-algorithm here 
by first calculating $\hat{\lambda}_j$ for each batch at E-step basically according to~\eqref{eq:e_step} 
but replacing $\lambda$ and $p_{\tau}$ by the value we learned during the training step.  
Then we can have
\beal{\label{eq:em_debiasing}
	\log L(\eta) &\geq \sum_{j=1}^{m_U} 
		\hat{\lambda}_j \biggl[ 
			\sum_{x_{jt1} \in X_{j}^1} \log \eta_{jt1}  + \sum_{x_{jt0} \in X_{j}^0} \log (1 - \eta_{jt0}) 
		\biggr] \nonumber \\
		& + \sum_{j=1}^{m_U} (1 - \hat{\lambda}_j) \bigl[\log \hat{p}_{\tau_j}  + \log P(X_{j}^1 \succ X_{j}^0) \bigr] 
}%
where the second term includes $\log P(X_{j}^1 \succ X_{j}^0)$, which is hard to optimize.  
We apply the idea of EM-algorithm again here.  
We use notation $R_j$ to represent $R(X_{j}^1, X_{j}^0)$.  
For each $\pi \in R_j$, we can calculate its conditional probability given $X_{j}^1 \succ X_{j}^0$, denoted as $\hat{q}_{\pi}$ by:
\beal{
	\hat{q}_{\pi} = P(\pi | X_{j}^1 \succ X_{j}^0; \hat{\eta}) = \frac{P(\pi; \hat{\eta})}{\sum_{\pi \in R_j}P(\pi; \hat{\eta})} 
}%
which is the E-step.  
According to Jensen's inequality we have:
\beal{\label{eq:semi_ranking_to_sum_of_ranking}
	\log P(X_1 \succ X_0) &= %\log \sum_{\substack{\pi: \forall x_{1} \in X_1, x_0 \in X_0, \\ \pi^{-1}(x_{0}) > \pi^{-1}(x_{1}) }} P(\pi)
	\log \sum_{\pi \in R_j} P(\pi) \nonumber \\
	&\geq \sum_{\pi \in R_j} \hat{q}_{\pi} \log P(\pi)
}%
where the last inequality yields the objective function we want to optimize.  
The correctness of EM-algorithm guarantees the convergence of optimizing this function.  


\hide{
We further utilzie Jensen's inequality to bound this term from bottom by
\beal{\label{eq:semi_ranking_to_sum_of_ranking}
	\log P(X_1 \succ X_0) &= %\log \sum_{\substack{\pi: \forall x_{1} \in X_1, x_0 \in X_0, \\ \pi^{-1}(x_{0}) > \pi^{-1}(x_{1}) }} P(\pi)
	\log \sum_{\pi \in R_j} P(\pi) \nonumber \\
	&\geq \frac{1}{|R_j|} \sum_{\pi \in R_j}  \log P(\pi)
}%
where 
}


Furthermore, according to the minorization-maximizaion algorithm used in~\cite{hunter:aos2004}, 
we can obtain the lower bound for $\log P(\pi)$, which is defined by the Plackett-Luce model, by:
\beal{\label{eq:mm_algorithm}
	\log P(\pi) &= \sum_{t=1}^{k-1} \biggl[ 
		\log \eta_{\pi^{-1}(t)} - \log \sum_{s=t}^{k} \eta_{\pi^{-1}(s)}
	\biggr] \nonumber \\
	&\geq \sum_{t=1}^{k-1} \biggl[ 
		\log \eta_{\pi^{-1}(t)} - \frac{\sum_{s=t}^{k} \eta_{\pi^{-1}(s)}}{\sum_{s=t}^{k} \hat{\eta}_{\pi^{-1}(s)}}
	\biggr]
}%
where $\hat{\eta}_i$ is the estimated parameter of last iteration.  



By combining \eqref{eq:em_debiasing},~\eqref{eq:semi_ranking_to_sum_of_ranking} and \eqref{eq:mm_algorithm}, 
we can obtain the objective function to optimize as:
\beal{\label{eq:objective_function}
	Q(\eta) &= \sum_{j=1}^{m_U} 
		\hat{\lambda}_j \biggl[ 
			\sum_{x_{jt1} \in X_{j}^1} \log \eta_{jt1}  + \sum_{x_{jt0} \in X_{j}^0} \log (1 - \eta_{jt0}) 
		\biggr] \nonumber \\
		&+ \sum_{j=1}^{m_U} (1 - \hat{\lambda}_j)  \sum_{\pi \in R_j}  \hat{q}_{\pi}
		\sum_{t=1}^{k-1} \biggl[ 
			\log \eta_{\pi^{-1}(t)} - \frac{\sum_{s=t}^{k} \eta_{\pi^{-1}(s)}}{\sum_{s=t}^{k} \hat{\eta}_{\pi^{-1}(s)}}
		\biggr]  
}%
Notice that $Q(\eta)$ is actually a lowerbound of the original log-likelihood function.  
Moreover, for two EM-step and one MM-step we apply in deriving $Q$-function, 
it is provent that by improving $Q(\eta)$ from this iteration $Q(\hat{\eta})$, 
we can achieve no less improvement on the loglikelihood.  
Therefore optimizing $Q(\eta)$ can also optimizes the loglikelihood.  



Take the derivative, we can obtain
\beal{\label{eq:derivative_of_objective_function}
    \frac{\partial Q(\eta)}{\partial \eta_i} &= 
        \sum_{j \in M_1(i)}\frac{1}{\eta_i}
        + \sum_{j' \in M_0(i)} \frac{ \hat{\lambda}_{j'}}{1 - \eta_i} \nonumber \\
        %&+ \sum_{j \in M_1(i)} \frac{1 - \hat{\lambda}_j}{\eta_i}
        &+ \sum_{j=1}^{m_U} (1 - \hat{\lambda}_j) \sum_{\pi \in R_j} \hat{q}_{\pi} \biggl[
            \sum_{t=1}^{|X_j^1|} \frac{\mathbf{1}_{\{\pi^{-1}(i) \geq t \}}}{
                \sum_{s=t}^{k} \hat{\eta}_{\pi^{-1}(s)}
            }
        \biggr]
}%
where $M_1(i)$ and $M_0(i)$ are defined as $M_y(i) = \{j:x_i \in X_j^y\}$ for $y \in \{0, 1\}$.  
The updating rule can be obtained by solving $\partial Q(\eta) / \partial \eta_i = 0$.  

By iteratively updating the scores to optimize the likelihood of the annotation on test data, 
we can obtain the inferred predictive distribution of each item.  
Based on which, we can determine the inferred binary label for each data item by assigning $y_i'=1$ if $\hat{\eta}_i > 0.5$, 
or $y_i'=0$ otherwise.  
Notice that we do not further tune the threshold in this step, 
as the scores we learned here are expected to be a reasonable estimate of the true $\eta_i$'s. 
Therefore, if the predictive distribution are known, 
learning theory guarantees that by using Bayes classifier (namely to take 0.5 as threshold) 
is supposed to yield the best expected performance.  





\section{Experimental Results}
\label{sec:exp}

In this section, we conduct experiments on a synthetic data set and a real data set 
to verify the effectiveness of our proposed worker model and debiasing technique.  

\subsection{Data sets}
We first introduce the data sets we used in this experiments.  
A summary of the data sets we used in our experiments are given in Table~\ref{tab:dataset}.  

\vpara{Synthetic.}
We construct synthetic data sets following the worker annotation model we proposed in Section~\ref{sec:worker}.
Suppose we have $n$ items in $X$, 
we first generate their predictive probability $\eta_i$ for each $x_i \in X$ 
from a Beta distribution $Beta(\alpha, \beta)$, 
then generate the true labels $Y$ by drawing $y_i$ from a Bernoulli distribution parameterized by $\eta_i$ for each $i$.  
In our synthetic data set, we set $\alpha = 2$ and $\beta = 4$ to simulate the case when negative data items overwhelm.

Then we generate $m$ batches of size $k$ by sampling without replacement for each batch.  
Notice that by ``without replacement'' we mean there are no identical data items within the same batch, 
while the same item can still appear in multiple batches 
as we do replace the items back into the pool after a batch is generated.  
Thereby we obtain the set of batches $B$.  
For each $b_j$ in $B$, we generate the workers' annotation $y_j'$ from our proposed batch annotation model.  
The probability of making independent judgments $\lambda$ is given.
The distribution of determining number of positive annotations $p_{\tau}$ is also assigned to be:
\beal{
	p_{\tau} \propto (\tau + 1)^{- \rho} \nonumber
}%
where $\rho$ is positive constant.  
In our synthetic data set, we set $\lambda = 0.5$ and $\rho = 2$.  

\vpara{Comments.}
We utilize a real world crowdsourcing data set.  
The original crowdsourced task is to identify inappropriate comments on LinkedIn posts published by companies or LinkedIn influencers.  
Inappropriate comments are defined as comments containing promotional, profane, blatant soliciting, random greeting comments, 
as well as comments with \emph{only} web links and contact information.  
In order to collect annotation of comments, 
for each post, $k$ comments are sampled and sent to CrowdFlower as a batch (unit).  
Workers are also provided with a codebook (instruction) explaining how to annotate the data.  
Each comment is regarded as a data item and can be annotated as positive (inappropriate comment) or negative (acceptable comment).  
Each batch is annotated by more than 5 workers.  

In order to provide test questions and track the performance of each worker, 
some of the batches are annotated by 9 trained LinkedIn employees (experts) with the same codebook and interface as used for crowd workers.  
The average Cohen's kappa for all expert pairs is 0.7881.  
For this experiments, we only adopt the batches with all of their data items annotated by both crowds and experts 
as we can use the experts' annotation as ground truth (aggregated by majority voting).  
Among which, 1,099 batches that are annotated before a worker actually works on the job 
are utilized as training data set $B_L$.  
while the other 5,267 batches are utilized as test data set $B_U$ to infer the 651 data items they covered.  


\begin{table}[!t]
\centering
 {\caption{Data set statistics.}\label{tab:dataset}}
{
  \begin{tabular}{c||c|c|c|c|c}
  \hline
  Data set & $k$ & $n_L$  &  $m_L$ & $n_U$  & $m_U$ \\ \hline \hline
 Synthetic & 5 & 1,000 & 10,000 & 5,000 & 50,000 \\ \hline 
 Comments  & 5 & 110   & 1,099  & 651   & 5,267  \\ \hline  
  \end{tabular}
}
\end{table}


\subsection{Experimental Setup}

\vpara{Comparing method.}
We compare the performance of our proposed method with several baselines:
\begin{itemize}
  \item \emph{Majority Voting (MV)}.
        For each data item in the test data set, 
        simply determine its inferred label by the way it is annotated by the majority of workers.  
        This aggregation strategy is adopted by many crowdsourcing service.  
  \item \emph{Plackett-Luce Model (PL)}.  
        A strategy is to fit the Plackett-Luce model on the test data by inferring the scores $s(x_i)$ associated with each data items.  
        We apply a Bayesian regularization on the inferred scores to confine it as $0 \leq s(x_i) \leq 1$.  
        We then infer a positive label to each data item with an inferred score $s(x_i) > 0.5$ and a negative label otherwise.  
  \item[*] \emph{Batch Annotation Model (BAM)}.  
        The debiasing strategy proposed in Section~\ref{sec:worker} and~\ref{sec:debias}.
\end{itemize}

\vpara{Evaluation.}
For baselines without training, we directly apply them on the test data set; 
for our proposed method, we first train the worker model on the training data set, 
then apply the debiasing strategy based on the trained worker model on the test data set.  
We compare the inferred labels to the ground-truth and evaluate the performance in terms of accuracy, precision, recall and $F_1$-score.  
As all the methods generate scores for each data item 
(for MV we take the proportion of workers who annotated a data item as positive as the inferred score),
we also evaluate the performance in terms of the area of ROC curves.  


\vpara{Experimental setup.}
For our proposed model, 
in training phase, we initialize $\lambda$ to be 0.5 and $p_{\tau}$ to be a random value between 0 and 1;
in debiasing phase, we initialize the all the inferred scores as $0.1$. 
For training the worker model, we set a fixed number of iteration as 100.  
Our experimental results presented later show the model converge within number of iterations much fewer than 100.  
For debiasing, we calculate the log-likelihood of the model and stop when the relative change of log-likelihood is within 0.001.  


\subsection{Experimental Results}


\vpara{Worker model fitting.}
We first verify the effectiveness of learning our proposed worker model.  
On synthetic data set, the ``true'' value of probability of making independent judgments $\lambda$ is set to 0.5.  
We learn the model from the synthetic training data and obtain the inferred $\hat{\lambda}$ as 0.4998, 
which reasonably recovers the original value.  
We also compare the original model parameters $p_{\tau}$'s to the inferred parameters in Figure~\ref{subfig:p_tau_synthetic}.  
The black dash line represents the original parameters used for generating synthetic annotation data, 
while the red solid line shows the inferred parameters of worker model, 
which seems as a precise fit of the original parameter.  
We also show the curve of log-likelihood of the training data set,
which seems to converge within 20 iterations.  

Given the effectiveness of our learning method verified, 
we apply the worker model trying to fit the data set of annotating inappropriate comments.  
The learned probability of a worker making independent judgments $\hat{\lambda}$ is 0.7877.  
The learned distribution for determining the number of positive annotation in a batch is presented in Figure~\ref{subfig:p_tau_lnkd}.  
It shows that a worker tends to annotate the entire batch as negative (\ie~acceptable comment) with a probability over 0.6, 
while picking only 1 of them as positive (\ie~inappropriate comment) also occurs with a relative high probability around 0.25.  
The workers seem to be reluctant to annotate more than 1 comments in a size-5 batch.  
This is coherent with most people's intuition that inappropriate comments are rare comparing to the entire set of comments.  

The convergence analysis is shown in Figure~\ref{subfig:convergence_lnkd}.  
The model converges within 50 iterations.  




\begin{figure}[!t]
  \centering
  \subfigure[Parameter comparison of $p_{\tau}$'s]{
    \label{subfig:p_tau_synthetic}
    \includegraphics[width=0.46\columnwidth]{figures/ptau_synthetic}
  }
  \subfigure[Convergence analysis]{
    \label{subfig:convergence_synthetic}
    \includegraphics[width=0.46\columnwidth]{figures/loglikelihood_synthetic}
  }
  \caption{\label{fig:synthetic_training}
  Learning worker model from the synthetic training data set.
  }
\end{figure}

\begin{figure}[!t]
  \centering
  \subfigure[Parameter analysis of $p_{\tau}$'s]{
    \label{subfig:p_tau_lnkd}
    \includegraphics[width=0.46\columnwidth]{figures/ptau_lnkd}
  }
  \subfigure[Convergence analysis]{
    \label{subfig:convergence_lnkd}
    \includegraphics[width=0.46\columnwidth]{figures/loglikelihood_lnkd}
  }
  \caption{\label{fig:lnkd_training}
  Learning worker model from the comments training data set.
  }
\end{figure}


\vpara{Performance comparison.}
We proceed to evaluate the performance of different aggregation strategies on both data sets.  
The overall performance results are shown in Table~\ref{tab:performance}.  
In both data sets, our proposed debiasing strategy is a clear winner in terms of $F_1$-score, 
and also achieves the best accuracies.  

In synthetic data set, majority voting fails to identify most of the positive data items, 
and therefore achieves an extremely low recall.  
Although in practices one can usually achieve a better performance by adjusting the threshold of assigning a positive label, 
the determination of the threshold is usually empirical without a clear guideline.  
PL-model, in contrast, achieves a relatively low precision.  
Our proposed method is proven to achieve the best overall performance in terms of $F_1$-score.  

In comments data set, majority voting again obtains a poor recall below 80\%, 
while our proposed method achieves a comparable precision and a higher recall of 87\%, 
and therefore beat all the other baselines in terms of $F_1$-score.  


It is worth noting that in both data sets, the AUC of all the methods are comparable.  
This implies the quality of ranking recovered by all the methods are similar.  
However, to obtain reliable annotations, it is still important to accurately infer the scales of scores.  

% The strategy of determining the threshold for differentiating positive and negative data items makes the difference.  


\begin{table}[!t]
\centering
 {\caption{Performance comparison of Majority Voting (MV), Plackett-Luce Model (PL) and Batch Annotation Model (BAM).  
 All results are shown as percents.}\label{tab:performance}}
{
  \begin{tabular}{@{}c@{}|@{}c@{}||c|c|c|c|c@{}}
    \hline
        Data set & Method    & Acc.           & Prc.           & Rcl.           & $F_1$          & AUC \\ \hline \hline
        \multirow{3}{*}{Synthetic}
                 & MV        & 83.04          & \textbf{98.91} & 00.10          & 17.67          & 93.84          \\ \cline{2-7}
                 & PL        & 82.74          & 52.25          & \textbf{92.75} & 66.85          & 94.46          \\ \cline{2-7}
                 & BAM       & \textbf{89.98} & 71.40          & 77.72          & \textbf{74.43} & \textbf{94.96} \\ \hline
        \multirow{3}{*}{Comments}
                 & MV        & 95.55          & \textbf{93.75} & 79.65          & 86.12          & 99.05          \\ \cline{2-7}
                 & PL        & 94.62          & 85.45          & 83.19          & 84.30          & 96.40          \\ \cline{2-7}
                 & BAM       & \textbf{96.77} & 93.40          & \textbf{87.61} & \textbf{90.41} & \textbf{99.06} \\ \hline
  \end{tabular}
}
\end{table}

%!TEX root=betamain.tex

\section{Extensions}
\label{sec:ext}

In this section, we discuss several straightforward extensions of our proposed worker model and debiasing strategies, 
with respect to some useful applications other than binary classification: 
rating estimation and multi-class classification.  
We also discuss how to extend our model to the more general case when different workers have different biases.  

\vpara{Rating estimation.}
In rating estimation,
each data item $x_i$ is no longer associated with a discrete label
from a finite set of labels, 
but instead, a real value $y_i \in \RR$.  
\hide{
Suppose workers are asked to provide ratings on each data item in batches, 
and our goal is to identify the true (unknown) rating of each data item.  
A na\"{i}ve way to aggregate their ratings on the same item 
is to take the average value of their ratings as the estimated rating.  
However, when multiple data items are grouped into batches, 
there can be bias similar to the one introduced in this paper.
}
Although we do not explicitly formalize our problem for a rating task, 
with some straightforward modifications, 
our techniques can still be applied 
if the workers are asked to rate data items in batches.  

Without loss of generality, we can assume $0 < y_i < \infty$.  
If the actual rating can be negative, 
we can always apply a certain sigmoid function to normalize the scores to be positive values.  
For independent judging, we can design a distribution \wrt~$y_i$ from where a worker draws a rating $y_i'$ for $x_i$,
%well-regularized distribution with expectation of $y_i$ 
%for a worker to draw a rating,
~\eg~a Gaussian distribution $\cN(y_i, 1)$.  
For relative judging, we can still assume that the worker generates a ranking from Plackett-Luce model with parameters $y_i$'s, 
and introduce another distributions \wrt~the ranking of each data item $\pi^{-1}(x_i)$
from where a worker generates the ratings.  
%for generating rating for each data item
%from a distribution \emph{only} depending on their ranking, 
%of data items at different positions in the ranking, 
%which can be learned from the training data.  
For example, $y_i' \sim \cN(\pi^{-1}(x_i), 1)$.  
%workers may tend to generate a rating from Gaussian distribution centered at $\mu_1=5.0$ for a top-ranked data item $\pi(1)$, 
%but generate a rating from another Gaussian distribution centered at $\mu_5=1.0$ for a data item ranked as the fifth $\pi(5)$.  
\hide{If the training data is sparse and the ratings can take on any real values, 
it is probably preferable to employ some parametric distributions.}
Once the design of model is accomplished, 
it is straightforward to apply the same technique described in this paper 
to derive the debiasing strategy.
% by maximizing the likelihood of observed annotations 
% to estimate the underlying ratings for unrated data items.  


\vpara{Multi-class classification.}
In a multi-class classification problem, 
the label set $\cY$ may contain more than $2$ possible labels.  
Workers are usually requested to assign data items with different labels.  
This is a natural extension from binary classification problem.  

If the labels in $\cY$ are ordinal, 
for example, judging whether a review is ``very helpful'', ``helpful'' or ``not helpful'', 
the problem reduces to a rating estimation problem, 
where the possible values of rating are discrete values.  
We can simply apply the extended strategy described above.  
If the labels in $\cY$ do not have an order, 
%but the task does allow each data item to be assigned multiple labels, 
the problem can be reduced to several binary classification problems, 
which is straightforward to apply our strategy for debiasing workers' annotations.  


\vpara{Personalized worker model.}
To address different behaviors of different workers, 
we may want to employ different worker model separately for different workers, 
instead of using an identical model for all the workers.  
This can be achieved by rewriting~\eqref{eq:log_l_eta} as 
\beal{\label{eq:log_l_eta_diff}
	\log L (\eta) = \sum_{w=1}^W
	\sum_{j=1}^{m_w}
	\log \biggl[
		&\hat{\lambda}_w %\prod_{t=1}^{k} \eta_{jt}^{y_{jt}} (1 - \eta_{jt})^{(1 - y_{jt})}
		\prod_{x_{jt_1} \in X_{w j}^1} \eta_{jt_1} \prod_{x_{jt_0} \in X_{w j}^0} (1 - \eta_{jt_0}) \nonumber \\
		+ &(1 - \hat{\lambda}_w) \hat{p}_{w \tau_j} P(X_{w j}^1 \succ X_{w j}^0)
	\biggr] \nonumber
}%
where each $w$ corresponds a worker and each worker annotates $m_w$ batches.  



% If the labels in $\cY$ do not have an order, 
% and it is required that each data item can only be assigned one label, 
% we can still apply multiple binary classification framework, but 


\section{Related Work}
\label{sec:related}

\vpara{Annotation bias.}
A number of studies have been conducted on evaluating data quality collected from crowds, and modeling annotator behaviors to optimize the data quality.
Snow~\etal~\cite{snow:emnlp2008} explored the performance with non-expert annotators in several NLP tasks.
Raykar~\etal~\cite{raykar:nips2011ranking,raykar:icml2009,raykar:jmlr2010}
studied how to learn a model with noisy labeling. Specifically, they employ a logistic regression classifier, and insert hidden
variables indicating whether an annotator tells the truth. %They utilize an EM algorithm to infer the real ground-truth while training the model.
Whitehill~\etal~\cite{whitehill:nips2009} modeled the annotator ability, data item difficulty, and inferred the true label from the crowds in a unified model.
Venanzi~\etal~\cite{venanzi:www2014} proposed a community-based label aggregation model to identify different types of workers, 
and correct their labels correspondingly.  
However, they do not study the case when data items are grouped into batches.
Das~\etal~\cite{das:kdd2013} addressed the interactions of opinions between people connected by networks.
Demeester~\etal~\cite{demeester:wsdm2014} discussed the disagreement between different users on assessment of web search results.
Their studies also focus on understanding annotator behavior, but none of them consider the case when multiple data items are organized into a batch.

Scholer~\etal~\cite{scholer:sigir2013,scholer:sigir2011} studied the annotation disagreements in 
a relevance assessment data set.  
%TREC data set, which is on relevance assessment tasks in information retrieval.  
They discovered the correlation between annotations of similar data items.  %, and the estimated time between the two annotation.  
They also explored ``threshold priming'' in annotation, 
where the annotators tend to make similar judgments or apply similar standard on consecutive data items they review. 
%tend to make similar judgments to consecutive data items.  
However, their studies focus more on qualitative conclusions, without a quantitative model to characterize and measure the discovered factors.  
Carterette~\etal~\cite{carterette:effect2010} provided several assessor models for the TREC data set.  
Mozer~\etal~\cite{mozer:nips2010} studied the similar ``relativity of
judgments'' phenomenon on sequential tasks instead of small batches.  
Their focus is more on data items presented as a long sequence, while we focus more on data items presented in small batches.  
%Also, they focused more on debiasing not active learning.  %but did not apply active learning based on the proposed models.   
%p


\hide{
\vpara{Batch active learning.}
Active learning has been extensively studied.
Settles~\etal~\cite{settles:2010survey} summarized a number of active learning strategies.
Batch active learning, in contrast to traditional active learning settings, aims to choose a set of data items to query, which proposes some unique challenges.
Some strategies focus on optimizing the helpfulness of a data batch.
Hoi~\etal~\cite{hoi:icml2006,hoi:www2006} utilized Fisher information matrix to choose the data items that are likely to change the classifier parameters most.
Brinker~\etal~\cite{brinker:icml2003} studied batch active learning for SVM and aimed to maximize the diversity within the selected set of data samples.
Guo~\etal~\cite{guo:nips2008} proposed discriminative active learning strategy by formulating the problem as an optimization problem.
% Azimi~\etal~\cite{azimi:icml2012} propose a Monte-Carlo method to sequentially select the batch of samples to query.
A number of strategies also aim at choosing the most representative data batch with regard to the unlabeled data set.
Yu~\etal~\cite{yu:icml2006} proposed a transductive experimental design, which prioritizes the data samples that represent the hard-to-predict data.
Chattopadhyay et al.~\cite{chattopadhyay:kdd2012} tried to choose the data batch to minimize Maximum Mean Discrepancy (MMD) to measure the difference in distribution between the labeled and unlabeled data.
There are studies addressing both intuitions.
Wang~\etal~\cite{wang:kdd2013} designed a framework to minimize the upper bound of empirical risk, which aims to find a batch of data items that are both discriminative and representative.
However, all of the above studies assume reliable oracles---which never are in multiple-annotator scenarios such as crowdsourcing.

\vpara{Active learning with crowds.}
Crowdsourcing serves as a potentially ideal oracle for active learning.
Two perspectives have been explored; 
first is how to select data items to query when the oracles are noisy.
Sheng~\etal~\cite{sheng:kdd2008} provided an empirical study on the performance of repeated labeling and developed an active learning strategy addressing both the ``label uncertainty'' and ``model uncertainty''.
Second is how to select annotators for crowdsourcing.
Donmez~\etal~\cite{donmez:kdd2009} studied this problem, by modeling the querying problem as a multi-armed bandit problem.  Each annotator is regarded as as a bandit, and a binary reward function is defined based on whether the oracle provides a correct label.  %However, the true label is unknown. They hence utilize majority voting to determine the ground truth.
Yan~\etal~\cite{yan:icml2011} explore both the problem of selecting query samples and selecting oracles, in context of a logistic regression classifier.
Kajino~\etal~\cite{kajino:aaai2012} proposed a convex optimization function for active learning in crowds.  
% vijayanarasimhan~\etal~\cite{vijayanarasimhan:cvpr2011} applied crowdsourcing to actively learn the detector for p
However, none of them leverages in-batch bias for active learning.
}



\section{Conclusion}
\label{sec:conclusion}

In this work we study a specific type of annotation bias in crowdsourcing, 
which occurs when data items are grouped into batches 
and submitted to workers to be judged simultaneously.  
We propose a novel worker model designed to capture this type of bias, 
and show how to train the worker model on annotation data.  
We also present how to debias the label obtained from crowds given a trained worker model.  
We conduct experiments on both synthetic data and real world data to verify the effectiveness of our methods.  

The observation of batch annotation bias might exist in many scenarios other than crowdsourcing, 
and therefore the debiasing strategy can trigger a broad range of applications. 
For example, the conference paper review system where each reviewer is assigned a batch of papers 
can also be regarded as a batch annotation.  

There are several interesting directions to extend this work.  
For example, one can extend the model to further incorporate the different behavior of each individual worker  
and adjust the debiasing strategy accordingly.  
Also, it would be interesting to see if it is possible to improve the efficiency of debiasing 
by actively assemble a batch of data items to collect the desired labels, 
instead of sending randomly formed batches to the crowds.  


\hide{
\vpara{Future work.}
There are several different ways of further strengthen and extending this work.  
\begin{enumerate}
  \item \emph{More data sets.}
        As we are unlikely to provide case studies on the current real-world data set we used,
        it would be interesting to collect another real-world data set with more details that can be disclosed.  
        It would also be necessary to check for multiple real world data sets to make sure the proposed method can be well generalized.  
  \item \emph{Individual workers.}
        We have not yet considered to learn the model for each individual worker.  
        One may argue that differences between individual workers can affect the performance, 
        when we assume all the workers follow the same model.  
        It is not hard to implement a version which treats each individual worker differently.  
        However, it might be challenging as training data for each individual worker can be extremely sparse.  
  \item \emph{Active learning.}
        Although active learning in this setting has been preliminarily explored in~\cite{zhuang:wsdm2015}, 
        it is still worth exploring active learning which aims to reduce label errors instead of improving classifier performance.  
        Active data annotation is probably a more accurate term.  
\end{enumerate} 
}


%\balance
%\clearpage
\vspace{-0.1in}
\small
\bibliographystyle{abbrv}

\bibliography{references}

%\input{appendix.tex}



\end{document}
