\section{Conclusion}
\label{sec:conclusion}

In this work we study a specific type of annotation bias in crowdsourcing, 
which occurs when data items are grouped into batches 
and submitted to workers to be judged simultaneously.  
We propose a novel worker model designed to capture this type of bias, 
and show how to train the worker model on annotation data.  
We also present how to debias the label obtained from crowds given a trained worker model.  
We conduct experiments on both synthetic data and real world data, % to verify the effectiveness of our methods.  
and observe that our proposed debiasing technique, 
with a fairly small training data set, 
can achieve up to +57\% $F_1$-score improvement in the synthetic data set 
compared to the na\"{i}ve majority voting strategy, 
and +17\% improvement over a tuned majority voting strategy.   
On a real data set, our technique can also achieve up to +15\% improvement on $F_1$-score
over the tuned majority voting strategy.  


The observation of batch annotation bias might exist in many scenarios other than crowdsourcing, 
and therefore the debiasing strategy can trigger a broad range of applications. 
For example, the conference paper review system where each reviewer is assigned a batch of papers 
can also be regarded as a batch annotation.  

There are several interesting directions to extend this work.  
For example, one can extend the model to further incorporate the different behavior of each individual worker  
and adjust the debiasing strategy accordingly.  
Also, it would be interesting to see if it is possible to improve the efficiency of debiasing 
by actively assemble a batch of data items to collect the desired labels, 
instead of sending randomly formed batches to the crowds.  


\hide{
\vpara{Future work.}
There are several different ways of further strengthen and extending this work.  
\begin{enumerate}
  \item \emph{More data sets.}
        As we are unlikely to provide case studies on the current real-world data set we used,
        it would be interesting to collect another real-world data set with more details that can be disclosed.  
        It would also be necessary to check for multiple real world data sets to make sure the proposed method can be well generalized.  
  \item \emph{Individual workers.}
        We have not yet considered to learn the model for each individual worker.  
        One may argue that differences between individual workers can affect the performance, 
        when we assume all the workers follow the same model.  
        It is not hard to implement a version which treats each individual worker differently.  
        However, it might be challenging as training data for each individual worker can be extremely sparse.  
  \item \emph{Active learning.}
        Although active learning in this setting has been preliminarily explored in~\cite{zhuang:wsdm2015}, 
        it is still worth exploring active learning which aims to reduce label errors instead of improving classifier performance.  
        Active data annotation is probably a more accurate term.  
\end{enumerate} 
}