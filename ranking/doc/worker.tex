\section{Worker Annotation Model On Small Batch of Data}
\label{sec:worker}

In this section, we first describe our model for workers' annotation behavior on a small batch of data;
then we introduce how to train the model based on a training data set.  
%finally we presents the parameters of this model learned from a real world data set.  

The basic idea is, when a worker judges a batch of data items, 
she can either choose to judge data items independently as if they are presented alone,
or to rank all the data items according to their relative relations 
and annotate the top several items as positive, leaving the rest in the batch as negative.  


\vpara{Plackett-Luce model.}
Before we delve in our model, we first recap a probability model for generating rankings 
based on scores associated with items, namely the Plackett-Luce model~\cite{luce:2005, plackett:1975}.  
Without loss of generality, suppose we are given a set of items $x_1, \ldots, x_k$.
Each item $x_i$ is associated with a certain score $s(x_i) > 0$.  
A ranking of these items can be represented as a bijection $\pi : \{i\}_{i = 1}^k \mapsto \{x_i\}_{i = 1}^k$, 
which maps the $i$-th position in the ranking to the item at this position.  
The corresponding ranking list can be represented as 
$\pi(1) \succ \cdots \succ \pi(k)$. 
In Plackett-Luce model, the probability of generating a ranking $\pi$ is:
\beal{
P(\pi) = \prod_{i = 1}^k \frac{s(x_i)}{\sum_{r = i}^{k} s(x_r)}	
}%
It can be interpreted as the following process:
Initially we have a pool $A$ of all the data items.  
Each time one picks an item $x_i$ from a pool $A$ of data items with a probability proportional to its score, namely:
\beal{
P(\text{picking $x_i$ from $A$}) = \frac{s(x_i)}{\sum_{x_r \in A} s(x_r)} \nonumber
}
This item is then removed from the pool $A$ and placed at the next position in the ranking.  
Repeat this operation until $A$ becomes empty.  
The probability of generating a ranking list according to this process is equivalent to the probability described in the Plackett-Luce model.  


\vpara{Worker model.}
Now we introduce our worker model for annotating small batches of data items.  
Again, without loss of generality, suppose we are given a batch $b_j$ of $(x_1, \ldots, x_k)$, namely $b_{ji} = i$.  
Also, recall that for each data item $x_i$, we denote its predictive probability $P(y_i = 1 | x_i)$ as $\eta_i$.  
Although this value is not explicitly known, 
we assume the worker has the ability to estimate this value based on the features of data item.  

When a worker starts to work on this batch of data items, 
with a certain probability $0 < \lambda < 1$, 
the worker chooses to judge each data item $x_i$ independently based on the $\eta_i$ value.  
More concretely, the worker generates $y_i' = 1$ with probability $\eta_i$ and $y_i' = 0$ with probability $(1 - \eta_i)$.  
With probability $(1 - \lambda)$, the worker chooses to first rank all the data items based on their probability of being positive, 
then annotates the top-$\tau$ items in the ranking as positive, leaving the other items annotated as negative.  
To be precise, the worker generates a ranking $\pi$ for $k$ items in the batch according to the Plackett-Luce model, 
with the scoring function defined as $s(x_i) = \eta_i$.  
Then the worker draws an integer $0 \leq \tau \leq k$ from a certain distribution, 
where $p_\tau$ denotes the probability of drawing the integer $\tau$.    
For data items $x_i \in \{\pi^{-1}(1), \ldots, \pi^{-1}(p)\}$ (could be empty if $p = 0$), the worker annotates them as $y_i'=1$, 
while other data items not within the top-$p$ of the ranking $\pi$ are annotated as $y_i' = 0$.

The intuition of this model is to capture two behavior pattern of workers.  
Sometimes workers use an absolute standard to judge data items from different batches, 
and therefore remain consistent performance.  
Nevertheless, sometimes workers might judge data items within a batch by comparison.  
They have an expectation of data distribution, 
which is reflected by the distribution of generating $\tau$, 
as it characterizes the probability of having $\tau$ positives within $k$ data items.   
They still have the ability to judge the relative relationships between data items in the same batch,
which is captured by the Plackett-Luce model for generating the ranking, 
but when they try to apply their expectation of data distribution on the batch, 
bias might occur.  

We summarize the process of generating annotation for a batch of data items in our proposed model as below:
\begin{enumerate}
  \item \label{step:tosscoin}
        Toss a coin $Z \sim Bernoulli(\lambda)$.  \\
        If $Z=1$, go to Step~\ref{step:ind};
        otherwise go to Step~\ref{step:rnk}.
  \item \label{step:ind}
        For each $x_i$, generate $y_i' \sim Bernoulli(\eta_i)$. \\
        Output the results and exit.  
  \item \label{step:rnk}
        Generate a ranking $\pi$ based on Plackett-Luce model for data items $x_i$ in the batch.
  \item \label{step:draw}
        Draw $\tau \sim Mult(p_{\tau})$.
  \item \label{step:annotate}
        For the top-$\tau$ items in ranking $\pi$, generate $y_i' = 1$; \\
        otherwise generate $y_i' = 0$.  \\
        Output the results and exit.  
\end{enumerate}



\vpara{Model learning.}
Parameters need to be determined in this worker model is:
the probability of making independent judgments $\lambda$, 
and the distribution of the number of positive annotation when making relative judgments, 
represented by $p_0, \ldots, p_k$, where $0 \leq p_{\tau} \leq 1$ and $\sum p_{\tau} = 1$.  

Suppose we are given a set of $n_L$ items $X_L$ with their predictive distribution $\eta_i$ known %true labels $Y_L$ known.  
Then we form them into $m_L$ batches $B_L$, send them to the crowds, 
and obtain their annotation from workers, denoted as $Y_L'$.  


For simplicity, we write $x_{b_{jt}}$ as $x_{jt}$, and similar to $y_{jt}$ and $\eta_{jt}$.
For each batch $b_j \in B_L$, we denote the set of items annotated by workers as positive as $X_{j}^1 = \{x_{jt} | y_{jt}' = 1\}$, 
and the set of items annotated as negative as $X_{j}^0 = \{x_{jt} | y_{jt}' = 0\}$.  


We train the model by maximum likelihood estimation.  
The likelihood of the obtained annotation can be written as:
\beal{
	L = \prod_{j=1}^{m_L} \biggl[ 
		\lambda \prod_{t=1}^{k} \eta_{jt}^{y_{jt}} (1 - \eta_{jt})^{(1 - y_{jt})} 
		%\prod_{x_{jt1} \in X_{j}^1} \eta_{jt1} \prod_{x_{jt0} \in X_{j}^0} (1 - \eta_{jt0}) \nonumber \\
		+ (1 - \lambda) p_{\tau_j} P(X_{j}^1 \succ X_{j}^0)  
	\biggr] 
}%
where $\tau_j = |X_{j}^1|$ is the number of positive annotation in batch $b_j$;
$P(X_{j}^1 \succ X_{j}^0)$ denotes the probability of generating any rankings $\pi$
that rank items in $X_{j}^1$ higher than any items in $X_{j}^0$, 
namely:
\beal{
	P(X_{j}^1 \succ X_{j}^0) = \sum_{\pi \in R(X_{j}^1, X_{j}^0)} P(\pi) \nonumber 
}%
where $R(X_1, X_0) = \{ \pi |\pi^{-1}(x_{0}) > \pi^{-1}(x_{1}),  \forall x_{1} \in X_1, x_0 \in X_0 \}$;
and $P(\pi)$ is defined by the Plackett-Luce model.  

Applying an EM-algorithm, where at E-step, we can have 
\beal{ \label{eq:e_step}
	\hat{\lambda}_j = \frac{\hat{\lambda} \prod_{t=1}^{k} \eta_{jt}^{y_{jt}} (1 - \eta_{jt})^{(1 - y_{jt})}}
	{\hat{\lambda} \prod_{t=1}^{k} \eta_{jt}^{y_{jt}} (1 - \eta_{jt})^{(1 - y_{jt})} + 
	(1 - \hat{\lambda}) \hat{p}_{\tau_j} P(X_{j}^1 \succ X_{j}^0)} 
}

And at M-step, we update the parameters $\hat{\lambda}$ and $\hat{p}_{\tau}$ by
\beal{
	\hat{\lambda} = \sum_{j=1}^{m_L} \hat{\lambda}_j, \hspace{0.2in}
	\hat{p}_{\tau} = \frac{1}{\hat{Z}} \sum_{j=1}^{m_L} (1 - \hat{\lambda}_j) \mathbf{1}_{\{ |X_{j}^1| = \tau\}} \nonumber 
}%
where $\hat{Z} = \sum_{j=1}^{m_L} (1 - \hat{\lambda}_j)$.  


However, in real data sets, we do not always know $\eta_i$ for each data item $x_i$, 
but only know their binary true labels $Y_L$.  
We can still learn the worker model based on this situation 
by substituting their labels $y_i \in \{0, 1\}$ into $\eta_i$.  
Directly substituting the value of 0 or 1 into $\eta_i$ may cause numerical problem in calculating $\lambda_j$.  
Therefore we regularize the score by using: 
\beal{
	\eta_i = \frac{y_i + \epsilon}{1 + 2 \epsilon} \nonumber
}%
where $\epsilon$ is a small constant which is set to $10^{-3}$ in our experiments.  





