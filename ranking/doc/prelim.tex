\section{Preliminaries}
\label{sec:prelim}

We start by introducing the background of a crowdsourcing platform, CrowdFlower.
Then we define the notations used in this paper and formalize the research problem we study.  

\subsection{Crowdsourcing Platform}

We briefly introduce the background of an online crowdsourcing platform, CrowdFlower\footnote{\url{http://www.crowdflower.com/}}.  
On CrowdFlower, users can design and submit a job to the online platform.
The platform will automatically assign the job to global crowd workers (termed as ``contributors'' in CrowdFlower).  
Each ``job'' submitted consists of a data set, an instruction of how the data set should be labeled 
and the interface designed for workers to interact with the platform.
A data set contains several ``units'', and each unit may be assigned to multiple workers to work on.  
Each worker can generate a ``judgment'' for each unit.  
Units are the minimum piece of work to be priced and assigned to workers.  
However, for some tasks the minimum unit price (\ie~0.01 USD) might be too high,
and it would be extremely inefficient for workers to judge only a data item in a unit, 
especially when the judgment needs to be done based on some context that potentially may be shared by different data items.  
For example, in the task of identifying whether a document is relevant to a certain query,
if each unit is designed to be a pair of document and query, 
a worker has to read both the query and the document to make a judgment;
in contrast, if a unit is designed to be a query and a batch of documents, 
the worker can read the query once, and make several judgments on multiple documents within a query.  
Thereby workers can save time, 
and users can lower their cost of data annotation.  


For each job, the user can require all the workers to go through a certain number of ``test questions'' 
before they actually start working on the job.  
Test questions are units of which the correct judgment is already known.
Therefore one can evaluate the accuracy of each worker.
Usually workers with an accuracy lower than 70\% on test questions are not allowed to work on the job.  
We believe that these test questions can be better utilized rather than simply calculating the overall accuracy.  
Richer information about worker behavior might be hidden within the judgments of test questions.  
Users can also insert some units with known answers intermingled with other units while workers work on the job,
to monitor the real time accuracy of each worker.  
Similarly, if the accuracy of a worker drops to lower than 70\%, she will be forbidden from working on the rest of the job.

After the judgments from (multiple) workers toward a unit is obtained, 
one can aggregate these judgments by a majority voting strategy to determine the final annotation of this unit.
However, this strategy may not be the best strategy due to the possible annotation bias when each unit contains multiple data items. 
We can also develop other aggregating strategies to apply on the judgments, 
which better recovers the true labels of data items.  


\subsection{Notation and Problem Definition}

First we need to formalize several concepts.  
Suppose we are given a set of data items $X = \{x_i\}$, where $i=1, \ldots, n$.  
Each data item is associated with a label $y_i \in \mathcal{Y}$,
where in a binary classification case, $\mathcal{Y} = \{0, 1\}$.  
Suppose each $(x_i, y_i)$ is generated from a joint probability distribution $P_{\mathcal{X} \mathcal{Y}}$.  
We define $\eta_i$ to be the conditional probability $P(y_i = 1 | x_i)$.  

In a job submitted to a crowdsourcing platform, 
we can assemble several data items into a batch (``unit'' in CrowdFlower terminology).  
Denote all the batches as a set of batches $B = \{b_j\}$ where $j = 1, \ldots, m$.  
Each batch $b_j$ can be represented by an ordered list of indices of data items in the batch, 
denoted as $(b_{j1}, \ldots, b_{jk})$.  
Thereby the batch $b_j$ consists of data items $(x_{b_{j1}}, \ldots, x_{b_{jk}})$.  
We do allow data items to appear in multiple batches in $B$.  
\hide{
Notice that we are only concerned with the scenario when data items are formed into \emph{small batches}.  
Depending on different format of data items, a small batch can have different scale of size.  
For example, in inappropriate comments identification task, we basically confine $k \leq 5$.   }
Also, for the sake of fully utilize the workforce of crowds, 
we only consider the scenario when all the batches in $B$ have the identical size $k$.  

As we assemble data items into small batch, 
each worker has to judge the entire batch as a single judgment.  
Given a batch $b_j$, the judgment provided by a worker can be represented as $y_j' = (y_{j1}', \ldots, y_{jk}')$, 
where $y_{j t}' \in \mathcal{Y}$ is the annotation of the $t$-th data item in $b_j$, given by the worker.  
Noting that the worker annotation $y_{j t}'$ can be different from the true label $y_{b_{j t}}$.  
We denote all the judgments as $Y'$.  
In real world crowdsourcing platform, a batch can actually be judged by multiple workers.  
We can simply regard them as judgments on multiple identical batches indexed differently, but with the same data items.  


Now we formalize the research problem as below:

\begin{problem} {Debiasing Annotation of Small Batches of Data.}
Given a small training set of data items $X_L$ as well as their true labels $Y_L$.  
Assemble these data items into batches $B_L$ and obtain the corresponding annotation from crowds $Y_L'$.  
Then with a larger set of data items $X_U$ without knowing their true labels, 
again assemble data items in $X_U$ into batches $B_U$, and obtain the corresponding annotation from crowds $Y'_U$.  
The goal is, based on the data and annotation collected from training data set 
and the noisy annotation for $X_U$ returned by the crowds, 
infer the true labels $Y_U$ for data items in $X_U$.  
\end{problem}


In this version of definition, we assume identical worker behavior.  
This assumption is not always true, 
but considering the data sparsity in real world crowdsourcing scenario 
(each worker usually answers only 10 test questions), 
building an accurate enough worker model for each worker is not realistic.  



