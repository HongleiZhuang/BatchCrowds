\section{Extensions}
\label{sec:ext}

In this section, we discuss two straightforward extensions of our proposed worker model and debiasing strategies, 
with respect to two useful applications other than binary classification: 
multi-class classification, and rating estimation.  

\vpara{Rating estimation.}
In a rating estimation,~\ie~a regression problem, 
each data item $x_i$ is no longer associated with a discrete label, 
but instead, a real value $y_i \in \RR$.  
For unrated data set, the true value of rating is unknown.  
Workers on crowdsourcing platform are usually asked to rate a data item.  
A naive way to aggregate their ratings on the same item 
is to take the average value of their ratings as the estimated rating.  
However, when multiple data items are grouped into batches, 
there can still be similar bias as we introduced in this paper.  

Although we do not explicitly formalize our problem for a rating task, 
with some straightforward modification, 
our strategy can still be applied.  
Without loss of generality, we can assume $0 < y_i < \infty$.  
If the actual rating can be negative, 
we can always apply a certain sigmoid function to normalize the scores to be positive values.  
For indepdent judging, we can design a well-regularized distribution with expectation of $y_i$ 
for a worker to draw rating,~\eg~Gaussian distribution centered at $y_i$.  
For relative judging, we can still assume the worker to generate a ranking from Plackett-Luce model with parameters $y_i$'s, 
and introduce distributions for generating rating of data items at different position in the ranking, 
which are to be learned from the training data.  
For example, workers may tend to generate a rating from Gaussian distribution centered at $\mu_1=5.0$ for a top-ranked data item $\pi(1)$, 
but generate a rating from another Gaussian distribution centered at $\mu_5=1.0$ for a data item ranked as the fifth $\pi(5)$.  
If the training data is sparse and rating can be any real values, 
it is probably preferred to employ some parametric distributions.    
Once the design of model is accomplished, 
it is straightforward to apply the same technique described in this paper 
to derive the debiasing strategy by maximizing the likelihood of observed annotations 
to estimate the underlying ratings for unrated data items.  


\vpara{Multi-class classification.}
In a multi-class classification problem, 
the label set $\cY$ may contain more than $2$ possible labels.  
Workers are usually requested to assign data items with different labels.  
This is a natural extension from binary classification problem.  

If the labels in $\cY$ have an order, 
for example, judging whether a review is ``very helpful'', ``helpful'' or ``not helpful'', 
the problem reduces to a rating estimation problem, 
where the possible value of rating are discrete values.  
We can simply apply the extended strategy described above.  

If the labels in $\cY$ do not have an order, 
%but the task does allow each data item to be assigned multiple labels, 
the problem can be reduces to several binary classification problems, 
which is straightforward to apply our strategy for debiasing workers' annotations.  

% If the labels in $\cY$ do not have an order, 
% and it is required that each data item can only be assigned one label, 
% we can still apply multiple binary classification framework, but 

